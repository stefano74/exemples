% Generated by Sphinx.
\def\sphinxdocclass{report}
\documentclass[letterpaper,10pt,french]{sphinxmanual}
\usepackage[utf8]{inputenc}
\DeclareUnicodeCharacter{00A0}{\nobreakspace}
\usepackage{cmap}
\usepackage[T1]{fontenc}
\usepackage{babel}
\usepackage{times}
\usepackage[Sonny]{fncychap}
\usepackage{longtable}
\usepackage{sphinx}
\usepackage{multirow}

\addto\captionsfrench{\renewcommand{\figurename}{Fig. }}
\addto\captionsfrench{\renewcommand{\tablename}{Tableau }}
\floatname{literal-block}{Code source }



\title{Client Test PyQt5 Documentation}
\date{25 March 2015}
\release{0.0.1}
\author{Stefano}
\newcommand{\sphinxlogo}{}
\renewcommand{\releasename}{Version}
\makeindex

\makeatletter
\def\PYG@reset{\let\PYG@it=\relax \let\PYG@bf=\relax%
    \let\PYG@ul=\relax \let\PYG@tc=\relax%
    \let\PYG@bc=\relax \let\PYG@ff=\relax}
\def\PYG@tok#1{\csname PYG@tok@#1\endcsname}
\def\PYG@toks#1+{\ifx\relax#1\empty\else%
    \PYG@tok{#1}\expandafter\PYG@toks\fi}
\def\PYG@do#1{\PYG@bc{\PYG@tc{\PYG@ul{%
    \PYG@it{\PYG@bf{\PYG@ff{#1}}}}}}}
\def\PYG#1#2{\PYG@reset\PYG@toks#1+\relax+\PYG@do{#2}}

\expandafter\def\csname PYG@tok@err\endcsname{\def\PYG@bc##1{\setlength{\fboxsep}{0pt}\fcolorbox[rgb]{1.00,0.00,0.00}{1,1,1}{\strut ##1}}}
\expandafter\def\csname PYG@tok@kr\endcsname{\let\PYG@bf=\textbf\def\PYG@tc##1{\textcolor[rgb]{0.00,0.44,0.13}{##1}}}
\expandafter\def\csname PYG@tok@ni\endcsname{\let\PYG@bf=\textbf\def\PYG@tc##1{\textcolor[rgb]{0.84,0.33,0.22}{##1}}}
\expandafter\def\csname PYG@tok@nc\endcsname{\let\PYG@bf=\textbf\def\PYG@tc##1{\textcolor[rgb]{0.05,0.52,0.71}{##1}}}
\expandafter\def\csname PYG@tok@sd\endcsname{\let\PYG@it=\textit\def\PYG@tc##1{\textcolor[rgb]{0.25,0.44,0.63}{##1}}}
\expandafter\def\csname PYG@tok@go\endcsname{\def\PYG@tc##1{\textcolor[rgb]{0.20,0.20,0.20}{##1}}}
\expandafter\def\csname PYG@tok@se\endcsname{\let\PYG@bf=\textbf\def\PYG@tc##1{\textcolor[rgb]{0.25,0.44,0.63}{##1}}}
\expandafter\def\csname PYG@tok@s\endcsname{\def\PYG@tc##1{\textcolor[rgb]{0.25,0.44,0.63}{##1}}}
\expandafter\def\csname PYG@tok@bp\endcsname{\def\PYG@tc##1{\textcolor[rgb]{0.00,0.44,0.13}{##1}}}
\expandafter\def\csname PYG@tok@gi\endcsname{\def\PYG@tc##1{\textcolor[rgb]{0.00,0.63,0.00}{##1}}}
\expandafter\def\csname PYG@tok@nn\endcsname{\let\PYG@bf=\textbf\def\PYG@tc##1{\textcolor[rgb]{0.05,0.52,0.71}{##1}}}
\expandafter\def\csname PYG@tok@ow\endcsname{\let\PYG@bf=\textbf\def\PYG@tc##1{\textcolor[rgb]{0.00,0.44,0.13}{##1}}}
\expandafter\def\csname PYG@tok@ss\endcsname{\def\PYG@tc##1{\textcolor[rgb]{0.32,0.47,0.09}{##1}}}
\expandafter\def\csname PYG@tok@kd\endcsname{\let\PYG@bf=\textbf\def\PYG@tc##1{\textcolor[rgb]{0.00,0.44,0.13}{##1}}}
\expandafter\def\csname PYG@tok@il\endcsname{\def\PYG@tc##1{\textcolor[rgb]{0.13,0.50,0.31}{##1}}}
\expandafter\def\csname PYG@tok@s1\endcsname{\def\PYG@tc##1{\textcolor[rgb]{0.25,0.44,0.63}{##1}}}
\expandafter\def\csname PYG@tok@ne\endcsname{\def\PYG@tc##1{\textcolor[rgb]{0.00,0.44,0.13}{##1}}}
\expandafter\def\csname PYG@tok@ge\endcsname{\let\PYG@it=\textit}
\expandafter\def\csname PYG@tok@sr\endcsname{\def\PYG@tc##1{\textcolor[rgb]{0.14,0.33,0.53}{##1}}}
\expandafter\def\csname PYG@tok@gr\endcsname{\def\PYG@tc##1{\textcolor[rgb]{1.00,0.00,0.00}{##1}}}
\expandafter\def\csname PYG@tok@cp\endcsname{\def\PYG@tc##1{\textcolor[rgb]{0.00,0.44,0.13}{##1}}}
\expandafter\def\csname PYG@tok@vi\endcsname{\def\PYG@tc##1{\textcolor[rgb]{0.73,0.38,0.84}{##1}}}
\expandafter\def\csname PYG@tok@nt\endcsname{\let\PYG@bf=\textbf\def\PYG@tc##1{\textcolor[rgb]{0.02,0.16,0.45}{##1}}}
\expandafter\def\csname PYG@tok@gh\endcsname{\let\PYG@bf=\textbf\def\PYG@tc##1{\textcolor[rgb]{0.00,0.00,0.50}{##1}}}
\expandafter\def\csname PYG@tok@mf\endcsname{\def\PYG@tc##1{\textcolor[rgb]{0.13,0.50,0.31}{##1}}}
\expandafter\def\csname PYG@tok@mo\endcsname{\def\PYG@tc##1{\textcolor[rgb]{0.13,0.50,0.31}{##1}}}
\expandafter\def\csname PYG@tok@gu\endcsname{\let\PYG@bf=\textbf\def\PYG@tc##1{\textcolor[rgb]{0.50,0.00,0.50}{##1}}}
\expandafter\def\csname PYG@tok@no\endcsname{\def\PYG@tc##1{\textcolor[rgb]{0.38,0.68,0.84}{##1}}}
\expandafter\def\csname PYG@tok@c1\endcsname{\let\PYG@it=\textit\def\PYG@tc##1{\textcolor[rgb]{0.25,0.50,0.56}{##1}}}
\expandafter\def\csname PYG@tok@kc\endcsname{\let\PYG@bf=\textbf\def\PYG@tc##1{\textcolor[rgb]{0.00,0.44,0.13}{##1}}}
\expandafter\def\csname PYG@tok@nd\endcsname{\let\PYG@bf=\textbf\def\PYG@tc##1{\textcolor[rgb]{0.33,0.33,0.33}{##1}}}
\expandafter\def\csname PYG@tok@na\endcsname{\def\PYG@tc##1{\textcolor[rgb]{0.25,0.44,0.63}{##1}}}
\expandafter\def\csname PYG@tok@mi\endcsname{\def\PYG@tc##1{\textcolor[rgb]{0.13,0.50,0.31}{##1}}}
\expandafter\def\csname PYG@tok@nl\endcsname{\let\PYG@bf=\textbf\def\PYG@tc##1{\textcolor[rgb]{0.00,0.13,0.44}{##1}}}
\expandafter\def\csname PYG@tok@k\endcsname{\let\PYG@bf=\textbf\def\PYG@tc##1{\textcolor[rgb]{0.00,0.44,0.13}{##1}}}
\expandafter\def\csname PYG@tok@m\endcsname{\def\PYG@tc##1{\textcolor[rgb]{0.13,0.50,0.31}{##1}}}
\expandafter\def\csname PYG@tok@cs\endcsname{\def\PYG@tc##1{\textcolor[rgb]{0.25,0.50,0.56}{##1}}\def\PYG@bc##1{\setlength{\fboxsep}{0pt}\colorbox[rgb]{1.00,0.94,0.94}{\strut ##1}}}
\expandafter\def\csname PYG@tok@nf\endcsname{\def\PYG@tc##1{\textcolor[rgb]{0.02,0.16,0.49}{##1}}}
\expandafter\def\csname PYG@tok@sh\endcsname{\def\PYG@tc##1{\textcolor[rgb]{0.25,0.44,0.63}{##1}}}
\expandafter\def\csname PYG@tok@nb\endcsname{\def\PYG@tc##1{\textcolor[rgb]{0.00,0.44,0.13}{##1}}}
\expandafter\def\csname PYG@tok@gt\endcsname{\def\PYG@tc##1{\textcolor[rgb]{0.00,0.27,0.87}{##1}}}
\expandafter\def\csname PYG@tok@w\endcsname{\def\PYG@tc##1{\textcolor[rgb]{0.73,0.73,0.73}{##1}}}
\expandafter\def\csname PYG@tok@vg\endcsname{\def\PYG@tc##1{\textcolor[rgb]{0.73,0.38,0.84}{##1}}}
\expandafter\def\csname PYG@tok@sc\endcsname{\def\PYG@tc##1{\textcolor[rgb]{0.25,0.44,0.63}{##1}}}
\expandafter\def\csname PYG@tok@cm\endcsname{\let\PYG@it=\textit\def\PYG@tc##1{\textcolor[rgb]{0.25,0.50,0.56}{##1}}}
\expandafter\def\csname PYG@tok@c\endcsname{\let\PYG@it=\textit\def\PYG@tc##1{\textcolor[rgb]{0.25,0.50,0.56}{##1}}}
\expandafter\def\csname PYG@tok@gs\endcsname{\let\PYG@bf=\textbf}
\expandafter\def\csname PYG@tok@o\endcsname{\def\PYG@tc##1{\textcolor[rgb]{0.40,0.40,0.40}{##1}}}
\expandafter\def\csname PYG@tok@s2\endcsname{\def\PYG@tc##1{\textcolor[rgb]{0.25,0.44,0.63}{##1}}}
\expandafter\def\csname PYG@tok@kt\endcsname{\def\PYG@tc##1{\textcolor[rgb]{0.56,0.13,0.00}{##1}}}
\expandafter\def\csname PYG@tok@gp\endcsname{\let\PYG@bf=\textbf\def\PYG@tc##1{\textcolor[rgb]{0.78,0.36,0.04}{##1}}}
\expandafter\def\csname PYG@tok@mb\endcsname{\def\PYG@tc##1{\textcolor[rgb]{0.13,0.50,0.31}{##1}}}
\expandafter\def\csname PYG@tok@mh\endcsname{\def\PYG@tc##1{\textcolor[rgb]{0.13,0.50,0.31}{##1}}}
\expandafter\def\csname PYG@tok@nv\endcsname{\def\PYG@tc##1{\textcolor[rgb]{0.73,0.38,0.84}{##1}}}
\expandafter\def\csname PYG@tok@sb\endcsname{\def\PYG@tc##1{\textcolor[rgb]{0.25,0.44,0.63}{##1}}}
\expandafter\def\csname PYG@tok@si\endcsname{\let\PYG@it=\textit\def\PYG@tc##1{\textcolor[rgb]{0.44,0.63,0.82}{##1}}}
\expandafter\def\csname PYG@tok@kp\endcsname{\def\PYG@tc##1{\textcolor[rgb]{0.00,0.44,0.13}{##1}}}
\expandafter\def\csname PYG@tok@vc\endcsname{\def\PYG@tc##1{\textcolor[rgb]{0.73,0.38,0.84}{##1}}}
\expandafter\def\csname PYG@tok@kn\endcsname{\let\PYG@bf=\textbf\def\PYG@tc##1{\textcolor[rgb]{0.00,0.44,0.13}{##1}}}
\expandafter\def\csname PYG@tok@gd\endcsname{\def\PYG@tc##1{\textcolor[rgb]{0.63,0.00,0.00}{##1}}}
\expandafter\def\csname PYG@tok@sx\endcsname{\def\PYG@tc##1{\textcolor[rgb]{0.78,0.36,0.04}{##1}}}

\def\PYGZbs{\char`\\}
\def\PYGZus{\char`\_}
\def\PYGZob{\char`\{}
\def\PYGZcb{\char`\}}
\def\PYGZca{\char`\^}
\def\PYGZam{\char`\&}
\def\PYGZlt{\char`\<}
\def\PYGZgt{\char`\>}
\def\PYGZsh{\char`\#}
\def\PYGZpc{\char`\%}
\def\PYGZdl{\char`\$}
\def\PYGZhy{\char`\-}
\def\PYGZsq{\char`\'}
\def\PYGZdq{\char`\"}
\def\PYGZti{\char`\~}
% for compatibility with earlier versions
\def\PYGZat{@}
\def\PYGZlb{[}
\def\PYGZrb{]}
\makeatother

\renewcommand\PYGZsq{\textquotesingle}

\begin{document}

\maketitle
\tableofcontents
\phantomsection\label{index::doc}


Contents:


\chapter{Module principale}
\label{main:module-principale}\label{main::doc}\label{main:module-main}\label{main:welcome-to-client-test-pyqt5-s-documentation}\index{main (module)}\index{FormArticle (classe dans main)}

\begin{fulllineitems}
\phantomsection\label{main:main.FormArticle}\pysiglinewithargsret{\strong{class }\code{main.}\bfcode{FormArticle}}{\emph{parent=None}, \emph{aSignal=None}, \emph{aLibelle='`}, \emph{aPrix=0}, \emph{aDate=\textless{}built-in method today of type object at 0x97f780\textgreater{}}}{}
fenêtre d'édition d'un article
utilisé pour ajouter/modifier un article
\index{AjouterArticle() (méthode main.FormArticle)}

\begin{fulllineitems}
\phantomsection\label{main:main.FormArticle.AjouterArticle}\pysiglinewithargsret{\bfcode{AjouterArticle}}{}{}
\end{fulllineitems}

\index{Annuler() (méthode main.FormArticle)}

\begin{fulllineitems}
\phantomsection\label{main:main.FormArticle.Annuler}\pysiglinewithargsret{\bfcode{Annuler}}{}{}
\end{fulllineitems}


\end{fulllineitems}

\index{MainWindow (classe dans main)}

\begin{fulllineitems}
\phantomsection\label{main:main.MainWindow}\pysiglinewithargsret{\strong{class }\code{main.}\bfcode{MainWindow}}{\emph{parent=None}, \emph{aConnection=None}}{}
fenêtre principale
permet de lister les articles
CRUD Articles
\index{AfficherFormulaire() (méthode main.MainWindow)}

\begin{fulllineitems}
\phantomsection\label{main:main.MainWindow.AfficherFormulaire}\pysiglinewithargsret{\bfcode{AfficherFormulaire}}{\emph{aLibelle='`}, \emph{aPrix=0}, \emph{aDate=datetime.date(2015}, \emph{3}, \emph{25)}}{}
\end{fulllineitems}

\index{AjouterArticle() (méthode main.MainWindow)}

\begin{fulllineitems}
\phantomsection\label{main:main.MainWindow.AjouterArticle}\pysiglinewithargsret{\bfcode{AjouterArticle}}{}{}
\end{fulllineitems}

\index{CommitSession() (méthode main.MainWindow)}

\begin{fulllineitems}
\phantomsection\label{main:main.MainWindow.CommitSession}\pysiglinewithargsret{\bfcode{CommitSession}}{}{}
\end{fulllineitems}

\index{ModifierArticle() (méthode main.MainWindow)}

\begin{fulllineitems}
\phantomsection\label{main:main.MainWindow.ModifierArticle}\pysiglinewithargsret{\bfcode{ModifierArticle}}{}{}
\end{fulllineitems}

\index{RemplirArticles() (méthode main.MainWindow)}

\begin{fulllineitems}
\phantomsection\label{main:main.MainWindow.RemplirArticles}\pysiglinewithargsret{\bfcode{RemplirArticles}}{}{}
\end{fulllineitems}

\index{RollbackSession() (méthode main.MainWindow)}

\begin{fulllineitems}
\phantomsection\label{main:main.MainWindow.RollbackSession}\pysiglinewithargsret{\bfcode{RollbackSession}}{}{}
\end{fulllineitems}

\index{SupprimerArticle() (méthode main.MainWindow)}

\begin{fulllineitems}
\phantomsection\label{main:main.MainWindow.SupprimerArticle}\pysiglinewithargsret{\bfcode{SupprimerArticle}}{}{}
\end{fulllineitems}

\index{afficherArticles() (méthode main.MainWindow)}

\begin{fulllineitems}
\phantomsection\label{main:main.MainWindow.afficherArticles}\pysiglinewithargsret{\bfcode{afficherArticles}}{}{}
\end{fulllineitems}

\index{changerProxy() (méthode main.MainWindow)}

\begin{fulllineitems}
\phantomsection\label{main:main.MainWindow.changerProxy}\pysiglinewithargsret{\bfcode{changerProxy}}{}{}
\end{fulllineitems}

\index{fermerAppli() (méthode main.MainWindow)}

\begin{fulllineitems}
\phantomsection\label{main:main.MainWindow.fermerAppli}\pysiglinewithargsret{\bfcode{fermerAppli}}{}{}
\end{fulllineitems}

\index{imprimer() (méthode main.MainWindow)}

\begin{fulllineitems}
\phantomsection\label{main:main.MainWindow.imprimer}\pysiglinewithargsret{\bfcode{imprimer}}{}{}
\end{fulllineitems}

\index{slot\_FormArticle\_closed() (méthode main.MainWindow)}

\begin{fulllineitems}
\phantomsection\label{main:main.MainWindow.slot_FormArticle_closed}\pysiglinewithargsret{\bfcode{slot\_FormArticle\_closed}}{\emph{alibelle}, \emph{aprix}, \emph{adate}}{}
\end{fulllineitems}


\end{fulllineitems}



\section{Proxy}
\label{proxy::doc}\label{proxy:proxy}
Ce module sert d'interface au serveur.
Toutes les méthodes d'accès aux modèles dans la base de données sont décrites ci-après.
\phantomsection\label{proxy:module-proxy}\index{proxy (module)}
Created on 18 mars 2015

@author: stefano

Module de connexion au serveur par type de proxy
\index{Proxy (classe dans proxy)}

\begin{fulllineitems}
\phantomsection\label{proxy:proxy.Proxy}\pysiglinewithargsret{\strong{class }\code{proxy.}\bfcode{Proxy}}{\emph{aConnection=None}, \emph{aApplication=None}}{}
classe d'interface Proxy
\index{afficherMainWindow() (méthode proxy.Proxy)}

\begin{fulllineitems}
\phantomsection\label{proxy:proxy.Proxy.afficherMainWindow}\pysiglinewithargsret{\bfcode{afficherMainWindow}}{}{}
Demande l'autorisation de l'affichage MainWindow
\begin{quote}\begin{description}
\item[{Type retourné}] \leavevmode
boolean

\end{description}\end{quote}

\end{fulllineitems}

\index{ajouterArticle() (méthode proxy.Proxy)}

\begin{fulllineitems}
\phantomsection\label{proxy:proxy.Proxy.ajouterArticle}\pysiglinewithargsret{\bfcode{ajouterArticle}}{\emph{alibelle}, \emph{aprix}, \emph{adate}}{}
Ajoute les champs d'un article
\begin{quote}\begin{description}
\item[{Paramètres}] \leavevmode\begin{itemize}
\item {} 
\textbf{\texttt{aid}} (\href{https://docs.python.org/library/functions.html\#int}{\emph{int}}) -- id de l'article à ajouter

\item {} 
\textbf{\texttt{alibelle}} (\href{https://docs.python.org/library/functions.html\#str}{\emph{str}}) -- libellé de l'article à ajouter

\item {} 
\textbf{\texttt{aprix}} (\href{https://docs.python.org/library/functions.html\#str}{\emph{str}}) -- prix de l'article à ajouter

\item {} 
\textbf{\texttt{adate}} (\href{https://docs.python.org/library/functions.html\#str}{\emph{str}}) -- date de l'article à ajouter

\end{itemize}

\end{description}\end{quote}

\end{fulllineitems}

\index{listerArticles() (méthode proxy.Proxy)}

\begin{fulllineitems}
\phantomsection\label{proxy:proxy.Proxy.listerArticles}\pysiglinewithargsret{\bfcode{listerArticles}}{}{}
Demande la liste des articles
\begin{quote}\begin{description}
\item[{Retourne}] \leavevmode
liste des articles

\item[{Type retourné}] \leavevmode
\href{https://docs.python.org/library/stdtypes.html\#dict}{dict}

\end{description}\end{quote}


\strong{Voir aussi:}


modifierArticle(), supprimerArticle(), ajouterArticle()



\begin{notice}{warning}{Avertissement:}
retourne tous les articles au format JSON pour serveurweb
\end{notice}

\begin{notice}{note}{Note:}
serveurweb retourne le dictionaire des articles sérialisé au format Django
\end{notice}

\begin{notice}{note}{À faire}

A implémenter dans les classe dérivant de Proxy
\end{notice}

\end{fulllineitems}

\index{modifierArticle() (méthode proxy.Proxy)}

\begin{fulllineitems}
\phantomsection\label{proxy:proxy.Proxy.modifierArticle}\pysiglinewithargsret{\bfcode{modifierArticle}}{\emph{aid}, \emph{alibelle}, \emph{aprix}, \emph{adate}}{}
Modifie les champs d'un article
\begin{quote}\begin{description}
\item[{Paramètres}] \leavevmode\begin{itemize}
\item {} 
\textbf{\texttt{aid}} (\href{https://docs.python.org/library/functions.html\#int}{\emph{int}}) -- id de l'article à modifier

\item {} 
\textbf{\texttt{alibelle}} (\href{https://docs.python.org/library/functions.html\#str}{\emph{str}}) -- libellé de l'article à modifier

\item {} 
\textbf{\texttt{aprix}} (\href{https://docs.python.org/library/functions.html\#str}{\emph{str}}) -- prix de l'article à modifier

\item {} 
\textbf{\texttt{adate}} (\href{https://docs.python.org/library/functions.html\#str}{\emph{str}}) -- date de l'article à modifier

\end{itemize}

\end{description}\end{quote}

\end{fulllineitems}

\index{supprimerArticle() (méthode proxy.Proxy)}

\begin{fulllineitems}
\phantomsection\label{proxy:proxy.Proxy.supprimerArticle}\pysiglinewithargsret{\bfcode{supprimerArticle}}{\emph{aid}}{}
Supprime les champs d'un article
\begin{quote}\begin{description}
\item[{Paramètres}] \leavevmode
\textbf{\texttt{aid}} (\href{https://docs.python.org/library/functions.html\#int}{\emph{int}}) -- id de l'article à supprimer

\end{description}\end{quote}

\end{fulllineitems}


\end{fulllineitems}

\index{ProxyREST (classe dans proxy)}

\begin{fulllineitems}
\phantomsection\label{proxy:proxy.ProxyREST}\pysiglinewithargsret{\strong{class }\code{proxy.}\bfcode{ProxyREST}}{\emph{aConnection=None}}{}
Classe Proxy pour serveurREST

\end{fulllineitems}

\index{ProxyWeb (classe dans proxy)}

\begin{fulllineitems}
\phantomsection\label{proxy:proxy.ProxyWeb}\pysiglinewithargsret{\strong{class }\code{proxy.}\bfcode{ProxyWeb}}{\emph{aConnection=None}}{}
Classe Proxy pour serveurweb

\end{fulllineitems}

\index{ProxyXMLRPC (classe dans proxy)}

\begin{fulllineitems}
\phantomsection\label{proxy:proxy.ProxyXMLRPC}\pysiglinewithargsret{\strong{class }\code{proxy.}\bfcode{ProxyXMLRPC}}{\emph{aConnection=None}}{}
Classe Proxy XMLRPC

\end{fulllineitems}



\chapter{Constantes}
\label{constantes::doc}\label{constantes:constantes}
Cette page représente la documentation des constantes.
\phantomsection\label{constantes:module-constantes}\index{constantes (module)}\index{ADR\_REST (dans le module constantes)}

\begin{fulllineitems}
\phantomsection\label{constantes:constantes.ADR_REST}\pysigline{\code{constantes.}\bfcode{ADR\_REST}\strong{ = `/serveurREST/'}}
URL du serveur REST

\end{fulllineitems}

\index{ADR\_WEB (dans le module constantes)}

\begin{fulllineitems}
\phantomsection\label{constantes:constantes.ADR_WEB}\pysigline{\code{constantes.}\bfcode{ADR\_WEB}\strong{ = `/serveurweb/'}}
URL du serveur web - serveur principale

\end{fulllineitems}

\index{ADR\_XMLRPC (dans le module constantes)}

\begin{fulllineitems}
\phantomsection\label{constantes:constantes.ADR_XMLRPC}\pysigline{\code{constantes.}\bfcode{ADR\_XMLRPC}\strong{ = `/serveurXMLRPC/'}}
URL du serveur XML-RPC

\end{fulllineitems}

\index{DEBUG (dans le module constantes)}

\begin{fulllineitems}
\phantomsection\label{constantes:constantes.DEBUG}\pysigline{\code{constantes.}\bfcode{DEBUG}\strong{ = True}}
Développement : DEBUG = True
Production : DEBUG = False

\end{fulllineitems}

\index{MODE\_ADD (dans le module constantes)}

\begin{fulllineitems}
\phantomsection\label{constantes:constantes.MODE_ADD}\pysigline{\code{constantes.}\bfcode{MODE\_ADD}\strong{ = `add'}}
MODE Ajout d'un objet

\end{fulllineitems}

\index{MODE\_MOD (dans le module constantes)}

\begin{fulllineitems}
\phantomsection\label{constantes:constantes.MODE_MOD}\pysigline{\code{constantes.}\bfcode{MODE\_MOD}\strong{ = `mod'}}
MODE Modification d'un objet

\end{fulllineitems}

\index{PROXY\_PWD (dans le module constantes)}

\begin{fulllineitems}
\phantomsection\label{constantes:constantes.PROXY_PWD}\pysigline{\code{constantes.}\bfcode{PROXY\_PWD}\strong{ = `guest'}}
Mot de passe

\end{fulllineitems}

\index{PROXY\_USER (dans le module constantes)}

\begin{fulllineitems}
\phantomsection\label{constantes:constantes.PROXY_USER}\pysigline{\code{constantes.}\bfcode{PROXY\_USER}\strong{ = `guest'}}
Utilisateur

\end{fulllineitems}



\chapter{Log}
\label{log::doc}\label{log:module-log}\label{log:log}\index{log (module)}
Created on 18 mars 2015

@author: stefano

Module de configuration des logs
\begin{quote}\begin{description}
\item[{Example}] \leavevmode
\end{description}\end{quote}

logger = logging.getLogger(\_\_name\_\_)

logger.debug(`trace de debug')
\index{RequireDebugFalse (classe dans log)}

\begin{fulllineitems}
\phantomsection\label{log:log.RequireDebugFalse}\pysiglinewithargsret{\strong{class }\code{log.}\bfcode{RequireDebugFalse}}{\emph{name='`}}{}
classe de filtre pour DEBUG = False

\end{fulllineitems}

\index{RequireDebugTrue (classe dans log)}

\begin{fulllineitems}
\phantomsection\label{log:log.RequireDebugTrue}\pysiglinewithargsret{\strong{class }\code{log.}\bfcode{RequireDebugTrue}}{\emph{name='`}}{}
classe de filtre pour DEBUG = True

\end{fulllineitems}

\index{configure() (dans le module log)}

\begin{fulllineitems}
\phantomsection\label{log:log.configure}\pysiglinewithargsret{\code{log.}\bfcode{configure}}{}{}
Configure les log avec dict\_log

\end{fulllineitems}


dictionnaire de configuration des logs \emph{dict\_log}:

\begin{Verbatim}[commandchars=\\\{\}]
\PYG{n}{dict\PYGZus{}log} \PYG{o}{=} \PYG{p}{\PYGZob{}}
    \PYG{l+s}{\PYGZsq{}}\PYG{l+s}{version}\PYG{l+s}{\PYGZsq{}}\PYG{p}{:} \PYG{l+m+mi}{1}\PYG{p}{,}
    \PYG{l+s}{\PYGZsq{}}\PYG{l+s}{disable\PYGZus{}existing\PYGZus{}loggers}\PYG{l+s}{\PYGZsq{}}\PYG{p}{:} \PYG{n+nb+bp}{False}\PYG{p}{,}
    \PYG{l+s}{\PYGZsq{}}\PYG{l+s}{filters}\PYG{l+s}{\PYGZsq{}}\PYG{p}{:} \PYG{p}{\PYGZob{}}
        \PYG{l+s}{\PYGZsq{}}\PYG{l+s}{require\PYGZus{}debug\PYGZus{}false}\PYG{l+s}{\PYGZsq{}}\PYG{p}{:} \PYG{p}{\PYGZob{}}
            \PYG{l+s}{\PYGZsq{}}\PYG{l+s}{()}\PYG{l+s}{\PYGZsq{}}\PYG{p}{:} \PYG{l+s}{\PYGZsq{}}\PYG{l+s}{log.RequireDebugFalse}\PYG{l+s}{\PYGZsq{}}\PYG{p}{,}
        \PYG{p}{\PYGZcb{}}\PYG{p}{,}
        \PYG{l+s}{\PYGZsq{}}\PYG{l+s}{require\PYGZus{}debug\PYGZus{}true}\PYG{l+s}{\PYGZsq{}}\PYG{p}{:} \PYG{p}{\PYGZob{}}
            \PYG{l+s}{\PYGZsq{}}\PYG{l+s}{()}\PYG{l+s}{\PYGZsq{}}\PYG{p}{:} \PYG{l+s}{\PYGZsq{}}\PYG{l+s}{log.RequireDebugTrue}\PYG{l+s}{\PYGZsq{}}\PYG{p}{,}
        \PYG{p}{\PYGZcb{}}\PYG{p}{,}
    \PYG{p}{\PYGZcb{}}\PYG{p}{,}
    \PYG{l+s}{\PYGZsq{}}\PYG{l+s}{formatters}\PYG{l+s}{\PYGZsq{}}\PYG{p}{:} \PYG{p}{\PYGZob{}}
        \PYG{l+s}{\PYGZsq{}}\PYG{l+s}{simple}\PYG{l+s}{\PYGZsq{}}\PYG{p}{:} \PYG{p}{\PYGZob{}}
            \PYG{l+s}{\PYGZsq{}}\PYG{l+s}{format}\PYG{l+s}{\PYGZsq{}}\PYG{p}{:} \PYG{l+s}{\PYGZsq{}}\PYG{l+s}{[}\PYG{l+s+si}{\PYGZpc{}(asctime)s}\PYG{l+s}{] }\PYG{l+s+si}{\PYGZpc{}(levelname)s}\PYG{l+s}{ }\PYG{l+s+si}{\PYGZpc{}(message)s}\PYG{l+s}{\PYGZsq{}}\PYG{p}{,}
        \PYG{l+s}{\PYGZsq{}}\PYG{l+s}{datefmt}\PYG{l+s}{\PYGZsq{}}\PYG{p}{:} \PYG{l+s}{\PYGZsq{}}\PYG{l+s}{\PYGZpc{}}\PYG{l+s}{Y\PYGZhy{}}\PYG{l+s}{\PYGZpc{}}\PYG{l+s}{m\PYGZhy{}}\PYG{l+s+si}{\PYGZpc{}d}\PYG{l+s}{ }\PYG{l+s}{\PYGZpc{}}\PYG{l+s}{H:}\PYG{l+s}{\PYGZpc{}}\PYG{l+s}{M:}\PYG{l+s}{\PYGZpc{}}\PYG{l+s}{S}\PYG{l+s}{\PYGZsq{}}
        \PYG{p}{\PYGZcb{}}\PYG{p}{,}
        \PYG{l+s}{\PYGZsq{}}\PYG{l+s}{verbose}\PYG{l+s}{\PYGZsq{}}\PYG{p}{:} \PYG{p}{\PYGZob{}}
            \PYG{l+s}{\PYGZsq{}}\PYG{l+s}{format}\PYG{l+s}{\PYGZsq{}}\PYG{p}{:} \PYG{l+s}{\PYGZsq{}}\PYG{l+s}{[}\PYG{l+s+si}{\PYGZpc{}(asctime)s}\PYG{l+s}{] }\PYG{l+s+si}{\PYGZpc{}(levelname)s}\PYG{l+s}{ [}\PYG{l+s+si}{\PYGZpc{}(name)s}\PYG{l+s}{.}\PYG{l+s+si}{\PYGZpc{}(funcName)s}\PYG{l+s}{:}\PYG{l+s+si}{\PYGZpc{}(lineno)d}\PYG{l+s}{] }\PYG{l+s+si}{\PYGZpc{}(message)s}\PYG{l+s}{\PYGZsq{}}\PYG{p}{,}
        \PYG{l+s}{\PYGZsq{}}\PYG{l+s}{datefmt}\PYG{l+s}{\PYGZsq{}}\PYG{p}{:} \PYG{l+s}{\PYGZsq{}}\PYG{l+s}{\PYGZpc{}}\PYG{l+s}{Y\PYGZhy{}}\PYG{l+s}{\PYGZpc{}}\PYG{l+s}{m\PYGZhy{}}\PYG{l+s+si}{\PYGZpc{}d}\PYG{l+s}{ }\PYG{l+s}{\PYGZpc{}}\PYG{l+s}{H:}\PYG{l+s}{\PYGZpc{}}\PYG{l+s}{M:}\PYG{l+s}{\PYGZpc{}}\PYG{l+s}{S}\PYG{l+s}{\PYGZsq{}}
        \PYG{p}{\PYGZcb{}}\PYG{p}{,}
    \PYG{p}{\PYGZcb{}}\PYG{p}{,}
    \PYG{l+s}{\PYGZsq{}}\PYG{l+s}{handlers}\PYG{l+s}{\PYGZsq{}}\PYG{p}{:} \PYG{p}{\PYGZob{}}
        \PYG{l+s}{\PYGZsq{}}\PYG{l+s}{null}\PYG{l+s}{\PYGZsq{}}\PYG{p}{:} \PYG{p}{\PYGZob{}}
            \PYG{l+s}{\PYGZsq{}}\PYG{l+s}{class}\PYG{l+s}{\PYGZsq{}}\PYG{p}{:} \PYG{l+s}{\PYGZsq{}}\PYG{l+s}{logging.NullHandler}\PYG{l+s}{\PYGZsq{}}\PYG{p}{,}
        \PYG{p}{\PYGZcb{}}\PYG{p}{,}
        \PYG{l+s}{\PYGZsq{}}\PYG{l+s}{console}\PYG{l+s}{\PYGZsq{}}\PYG{p}{:} \PYG{p}{\PYGZob{}}
            \PYG{l+s}{\PYGZsq{}}\PYG{l+s}{level}\PYG{l+s}{\PYGZsq{}}\PYG{p}{:} \PYG{l+s}{\PYGZsq{}}\PYG{l+s}{DEBUG}\PYG{l+s}{\PYGZsq{}}\PYG{p}{,}
            \PYG{l+s}{\PYGZsq{}}\PYG{l+s}{filters}\PYG{l+s}{\PYGZsq{}}\PYG{p}{:} \PYG{p}{[}\PYG{l+s}{\PYGZsq{}}\PYG{l+s}{require\PYGZus{}debug\PYGZus{}true}\PYG{l+s}{\PYGZsq{}}\PYG{p}{]}\PYG{p}{,}
            \PYG{l+s}{\PYGZsq{}}\PYG{l+s}{class}\PYG{l+s}{\PYGZsq{}}\PYG{p}{:} \PYG{l+s}{\PYGZsq{}}\PYG{l+s}{logging.StreamHandler}\PYG{l+s}{\PYGZsq{}}\PYG{p}{,}
            \PYG{l+s}{\PYGZsq{}}\PYG{l+s}{formatter}\PYG{l+s}{\PYGZsq{}}\PYG{p}{:} \PYG{l+s}{\PYGZsq{}}\PYG{l+s}{simple}\PYG{l+s}{\PYGZsq{}}
        \PYG{p}{\PYGZcb{}}\PYG{p}{,}
        \PYG{l+s}{\PYGZsq{}}\PYG{l+s}{development\PYGZus{}logfile}\PYG{l+s}{\PYGZsq{}}\PYG{p}{:} \PYG{p}{\PYGZob{}}
            \PYG{l+s}{\PYGZsq{}}\PYG{l+s}{level}\PYG{l+s}{\PYGZsq{}}\PYG{p}{:} \PYG{l+s}{\PYGZsq{}}\PYG{l+s}{DEBUG}\PYG{l+s}{\PYGZsq{}}\PYG{p}{,}
            \PYG{l+s}{\PYGZsq{}}\PYG{l+s}{filters}\PYG{l+s}{\PYGZsq{}}\PYG{p}{:} \PYG{p}{[}\PYG{l+s}{\PYGZsq{}}\PYG{l+s}{require\PYGZus{}debug\PYGZus{}true}\PYG{l+s}{\PYGZsq{}}\PYG{p}{]}\PYG{p}{,}
            \PYG{l+s}{\PYGZsq{}}\PYG{l+s}{class}\PYG{l+s}{\PYGZsq{}}\PYG{p}{:} \PYG{l+s}{\PYGZsq{}}\PYG{l+s}{logging.handlers.RotatingFileHandler}\PYG{l+s}{\PYGZsq{}}\PYG{p}{,}
            \PYG{l+s}{\PYGZsq{}}\PYG{l+s}{filename}\PYG{l+s}{\PYGZsq{}}\PYG{p}{:} \PYG{l+s}{\PYGZsq{}}\PYG{l+s}{/tmp/clientPyQt\PYGZus{}dev.log}\PYG{l+s}{\PYGZsq{}}\PYG{p}{,}
            \PYG{l+s}{\PYGZsq{}}\PYG{l+s}{maxBytes}\PYG{l+s}{\PYGZsq{}}\PYG{p}{:} \PYG{l+m+mi}{1024}\PYG{o}{*}\PYG{l+m+mi}{1024}\PYG{o}{*}\PYG{l+m+mi}{5}\PYG{p}{,} \PYG{c}{\PYGZsh{} 5 MB}
            \PYG{l+s}{\PYGZsq{}}\PYG{l+s}{backupCount}\PYG{l+s}{\PYGZsq{}} \PYG{p}{:} \PYG{l+m+mi}{7}\PYG{p}{,}
            \PYG{l+s}{\PYGZsq{}}\PYG{l+s}{formatter}\PYG{l+s}{\PYGZsq{}}\PYG{p}{:} \PYG{l+s}{\PYGZsq{}}\PYG{l+s}{verbose}\PYG{l+s}{\PYGZsq{}}
        \PYG{p}{\PYGZcb{}}\PYG{p}{,}
        \PYG{l+s}{\PYGZsq{}}\PYG{l+s}{production\PYGZus{}logfile}\PYG{l+s}{\PYGZsq{}}\PYG{p}{:} \PYG{p}{\PYGZob{}}
            \PYG{l+s}{\PYGZsq{}}\PYG{l+s}{level}\PYG{l+s}{\PYGZsq{}}\PYG{p}{:} \PYG{l+s}{\PYGZsq{}}\PYG{l+s}{ERROR}\PYG{l+s}{\PYGZsq{}}\PYG{p}{,}
            \PYG{l+s}{\PYGZsq{}}\PYG{l+s}{filters}\PYG{l+s}{\PYGZsq{}}\PYG{p}{:} \PYG{p}{[}\PYG{l+s}{\PYGZsq{}}\PYG{l+s}{require\PYGZus{}debug\PYGZus{}false}\PYG{l+s}{\PYGZsq{}}\PYG{p}{]}\PYG{p}{,}
            \PYG{l+s}{\PYGZsq{}}\PYG{l+s}{class}\PYG{l+s}{\PYGZsq{}}\PYG{p}{:} \PYG{l+s}{\PYGZsq{}}\PYG{l+s}{logging.handlers.RotatingFileHandler}\PYG{l+s}{\PYGZsq{}}\PYG{p}{,}
            \PYG{l+s}{\PYGZsq{}}\PYG{l+s}{filename}\PYG{l+s}{\PYGZsq{}}\PYG{p}{:} \PYG{l+s}{\PYGZsq{}}\PYG{l+s}{/tmp/clientPyQt\PYGZus{}prod.log}\PYG{l+s}{\PYGZsq{}}\PYG{p}{,}
            \PYG{l+s}{\PYGZsq{}}\PYG{l+s}{maxBytes}\PYG{l+s}{\PYGZsq{}}\PYG{p}{:} \PYG{l+m+mi}{1024}\PYG{o}{*}\PYG{l+m+mi}{1024}\PYG{o}{*}\PYG{l+m+mi}{5}\PYG{p}{,} \PYG{c}{\PYGZsh{} 5 MB}
            \PYG{l+s}{\PYGZsq{}}\PYG{l+s}{backupCount}\PYG{l+s}{\PYGZsq{}} \PYG{p}{:} \PYG{l+m+mi}{7}\PYG{p}{,}
            \PYG{l+s}{\PYGZsq{}}\PYG{l+s}{formatter}\PYG{l+s}{\PYGZsq{}}\PYG{p}{:} \PYG{l+s}{\PYGZsq{}}\PYG{l+s}{simple}\PYG{l+s}{\PYGZsq{}}
        \PYG{p}{\PYGZcb{}}\PYG{p}{,}
    \PYG{p}{\PYGZcb{}}\PYG{p}{,}
    \PYG{l+s}{\PYGZsq{}}\PYG{l+s}{loggers}\PYG{l+s}{\PYGZsq{}}\PYG{p}{:} \PYG{p}{\PYGZob{}}
                \PYG{l+s}{\PYGZsq{}}\PYG{l+s}{py.warnings}\PYG{l+s}{\PYGZsq{}}\PYG{p}{:} \PYG{p}{\PYGZob{}}
                    \PYG{l+s}{\PYGZsq{}}\PYG{l+s}{handlers}\PYG{l+s}{\PYGZsq{}}\PYG{p}{:} \PYG{p}{[}\PYG{l+s}{\PYGZsq{}}\PYG{l+s}{console}\PYG{l+s}{\PYGZsq{}}\PYG{p}{,} \PYG{l+s}{\PYGZsq{}}\PYG{l+s}{development\PYGZus{}logfile}\PYG{l+s}{\PYGZsq{}}\PYG{p}{,}\PYG{p}{]}\PYG{p}{,}
                \PYG{p}{\PYGZcb{}}\PYG{p}{,}
                \PYG{l+s}{\PYGZsq{}}\PYG{l+s}{\PYGZsq{}}\PYG{p}{:} \PYG{p}{\PYGZob{}}
                     \PYG{l+s}{\PYGZsq{}}\PYG{l+s}{handlers}\PYG{l+s}{\PYGZsq{}}\PYG{p}{:} \PYG{p}{[}\PYG{l+s}{\PYGZsq{}}\PYG{l+s}{console}\PYG{l+s}{\PYGZsq{}}\PYG{p}{,} \PYG{l+s}{\PYGZsq{}}\PYG{l+s}{production\PYGZus{}logfile}\PYG{l+s}{\PYGZsq{}}\PYG{p}{,} \PYG{l+s}{\PYGZsq{}}\PYG{l+s}{development\PYGZus{}logfile}\PYG{l+s}{\PYGZsq{}}\PYG{p}{]}\PYG{p}{,}
                     \PYG{l+s}{\PYGZsq{}}\PYG{l+s}{level}\PYG{l+s}{\PYGZsq{}}\PYG{p}{:} \PYG{l+s}{\PYGZdq{}}\PYG{l+s}{DEBUG}\PYG{l+s}{\PYGZdq{}}\PYG{p}{,}
                \PYG{p}{\PYGZcb{}}\PYG{p}{,}
    \PYG{p}{\PYGZcb{}}
\PYG{p}{\PYGZcb{}}
\end{Verbatim}


\chapter{Indices and tables}
\label{index:indices-and-tables}\begin{itemize}
\item {} 
\DUspan{xref,std,std-ref}{genindex}

\item {} 
\DUspan{xref,std,std-ref}{modindex}

\item {} 
\DUspan{xref,std,std-ref}{search}

\end{itemize}


\renewcommand{\indexname}{Index des modules Python}
\begin{theindex}
\def\bigletter#1{{\Large\sffamily#1}\nopagebreak\vspace{1mm}}
\bigletter{c}
\item {\texttt{constantes}}, \pageref{constantes:module-constantes}
\indexspace
\bigletter{l}
\item {\texttt{log}}, \pageref{log:module-log}
\indexspace
\bigletter{m}
\item {\texttt{main}}, \pageref{main:module-main}
\indexspace
\bigletter{p}
\item {\texttt{proxy}}, \pageref{proxy:module-proxy}
\end{theindex}

\renewcommand{\indexname}{Index}
\printindex
\end{document}
